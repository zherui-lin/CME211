\documentclass[12pt,letterpaper]{article}
\usepackage[letterpaper, total={6in, 8in}]{geometry}
\title{CME 211 Homework 4}
\author{Zherui Lin}
\begin{document}
\paragraph{} CME 211 Homework 4
\paragraph{} Zherui Lin
\paragraph{} An Truss object is fully declared by the joints file and the beams file of the given pathnames with abstraction. 
\paragraph{} It has 12 attributes: joint\_ids, joint\_coords, joint\_fxys, joint\_zerodisps, n\_joints, n\_zerodisps, beam\_ids, beam\_joints, joints\_beams\_other, n\_beams, matrix, and forces. It has 7 methods: \_\_init\_\_(joints\_file, beams\_file), handle\_joints(joints\_file), handle\_beams(beams\_file), generate\_matrix(), calculate\_forces(), PlotGeometry(plot\_file), and \_\_repr\_\_(). 
\paragraph{} Each method has its own task of assignments to attributes, and its implementation will not affect other methods, which includes decomposition and encapsulation:
\begin{enumerate}
\item \_\_init\_\_(joints\_file, beams\_file): The initialization method will call other methods except PlotGeometry(plot\_file), and \_\_repr\_\_(). All attributes will be assigned during this execution.
\item handle\_joints(joints\_file): There are 6 attributes being assigned in this method: joint\_ids, joint\_coords, joint\_fxys, joint\_zerodisps, n\_joints, n\_zerodisps, where joint\_ids and joint\_zerodisps are lists, and joint\_coords and joint\_fxys are dicts. Data in the file of the given path will be parsed and stored into these data structures.
\item handle\_beams(beams\_file): There are 4 attributes being assigned in this method: beam\_ids, beam\_joints, joints\_beams\_other, n\_beams, where beam\_ids is a list, and beam\_joints and joints\_beams\_other are dicts. Similarly, data in the file of the given path will be parsed and stored into these data structures.
\item generate\_matrix(): Only matrix attribute is assigned in this method, which is generated in CSR format from the attributes mentioned above. The value and positions of each elem in the matrix can be calculated from the attributes above. If the generated matrix is not square, the program will raise a runtime error since the matrix equation is not valid.
\item calculate\_forces(): Only forces attribute is assigned in this method, which is a numpy array of the beam forces result in order. The vector of the right hand side of the matrix equation is generated from the attributes above. Thus we can now try to solve the matrix equation by the matrix and the vector. If the equation is singular, a runtime error is raised, otherwise the forces can be generated by slicing the solution.
\item PlotGeometry(plot\_file): A figure of the beam geometry can be generated by beam\_joints attribute, and it is saved to the given pathname.
\item \_\_repr\_\_(): The return representation string is generated by the forces attribute mentioned above.
\end{enumerate}
\end{document}